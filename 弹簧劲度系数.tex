\documentclass[a4paper,12pt]{article}
\usepackage[top=1in, bottom=1in, left=1in, right=1in]{geometry}
\usepackage{ctex}
\usepackage{graphicx}
\usepackage{amsmath}
\usepackage{booktabs}

\title{弹簧劲度系数测量实验报告}
\author{}
\date{}

\begin{document}

\maketitle

\section{摘要}
本实验通过悬挂不同质量砝码的方法,使用弹簧、砝码组、刻度尺等仪器测量弹簧在不同拉力作用下的伸长量。根据胡克定律,通过绘制弹簧拉力$F$与伸长量$x$的关系图,并采用最小二乘法进行线性拟合,计算得到了弹簧的劲度系数$k$。实验结果表明,在弹性限度内弹簧的伸长量与拉力成正比,验证了胡克定律的正确性。

\section{实验原理}
根据胡克定律,在弹性限度内,弹簧的伸长量$x$与所受拉力$F$成正比,其数学表达式为:
\begin{equation}
    F = kx
\end{equation}
其中$k$为弹簧的劲度系数,单位为N/m。

实验通过悬挂不同质量$m$的砝码对弹簧施加拉力$F=mg$($g$为重力加速度),并测量对应的伸长量$x$。通过测量多组$(F,x)$数据,绘制$F-x$关系图,理论上应得到一条过原点的直线,其斜率即为弹簧的劲度系数$k$。

采用最小二乘法进行线性拟合,设拟合直线为$F = kx + b$,其中理论上截距$b$应为0。通过最小化残差平方和,可求得最佳拟合参数:
\begin{equation}
    k = \frac{n\sum (F_ix_i) - \sum F_i \sum x_i}{n\sum x_i^2 - (\sum x_i)^2}
\end{equation}
其中$n$为数据点个数。

\section{实验仪器及实验步骤}
\subsection{实验仪器}
\begin{itemize}
    \item 螺旋弹簧(待测)
    \item 砝码组(质量分别为50g、100g、150g、200g、250g)
    \item 刻度尺(最小刻度1mm)
    \item 铁架台
    \item 指针标记
\end{itemize}

\subsection{实验步骤}
\begin{enumerate}
    \item 安装实验装置:将弹簧竖直悬挂在铁架台上,下端安装指针标记,旁边固定刻度尺。
    \item 测量原长:记录弹簧未悬挂砝码时指针在刻度尺上的初始位置$l_0$。
    \item 逐次加载砝码:依次悬挂50g、100g、150g、200g、250g砝码,待稳定后分别记录指针位置$l_i$。
    \item 计算伸长量:用$x_i = l_i - l_0$计算各拉力下的伸长量。
    \item 重复测量:卸载砝码,重复步骤3-4两次,取伸长量平均值。
    \item 数据处理:绘制$F-x$关系图,用最小二乘法计算劲度系数$k$。
    \item 误差分析:讨论实验过程中可能的误差来源及其影响。
\end{enumerate}

\section{实验数据处理}
实验测量数据如下表所示(单位:拉力 - N,伸长量 - cm):

\begin{table}[h]
\centering
\caption{弹簧拉力与伸长量测量数据}
\begin{tabular}{|c|c|c|}
\hline
序号 & 拉力F(N) & 伸长量x(cm) \\ \hline
1 & 1.0 & 2.0 \\ \hline
2 & 1.9 & 4.1 \\ \hline
3 & 3.0 & 6.0 \\ \hline
4 & 4.2 & 7.9 \\ \hline
5 & 5.0 & 10.0 \\ \hline
\end{tabular}
\end{table}

弹簧原长为10.0 cm。根据胡克定律$F = kx$,我们绘制了拉力F与伸长量x的关系图,并使用最小二乘法进行线性拟合,结果如图1所示。

\begin{figure}[h]
\centering
\includegraphics[width=0.8\linewidth]{plots/F_x_curve.png}
\caption{弹簧拉力与伸长量关系图}
\label{fig:F-x}
\end{figure}

拟合得到的直线方程为:
$$ F = 0.519x - 0.096 $$

由此可得弹簧的劲度系数:
$$ k = 0.519\ \text{N/cm} = 51.9\ \text{N/m} $$

\section{分析讨论}
1. 误差分析:
\begin{itemize}
    \item 系统误差:可能来源于弹簧本身的质量、测量仪器的精度限制等
    \item 随机误差:读数时的视觉误差、弹簧振动等环境因素
    \item 拟合误差:最小二乘法拟合的残差为0.096 N,表明数据与线性模型存在一定偏差
\end{itemize}

2. 结果讨论:
\begin{itemize}
    \item 实验得到的劲度系数k=51.9 N/m
    \item 拟合直线的截距(-0.096 N)接近零,说明弹簧在初始状态基本不受力,符合理论预期
    \item 数据点与拟合直线吻合较好,验证了胡克定律的正确性
\end{itemize}

3. 改进建议:
\begin{itemize}
    \item 增加测量数据点数量以提高精度
    \item 使用更高精度的测量仪器
    \item 控制环境温度等可能影响弹簧性能的因素
\end{itemize}

\end{document}