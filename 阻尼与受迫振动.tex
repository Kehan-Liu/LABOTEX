\documentclass[UTF8]{ctexart}
\usepackage{geometry}
\geometry{a4paper, margin=1in}
\usepackage{amsmath}
\usepackage{graphicx}
\usepackage{float}
\usepackage{enumitem}
\usepackage{booktabs}

\title{阻尼与受迫振动}
\author{}
\date{}

\begin{document}

\maketitle

\section{摘要}
本实验使用波耳共振仪研究了阻尼振动和受迫振动的特性。通过调整永久磁铁的位置改变阻尼系数,测量了不同阻尼状态下摆轮的振幅衰减规律,计算了阻尼系数$\beta$和品质因数$Q$。在受迫振动实验中,通过步进电机施加周期性外力,测量了系统的幅频特性和相频特性,观察了共振现象。实验结果表明:阻尼振动振幅随时间呈指数衰减,衰减速率与阻尼系数相关;受迫振动的稳态振幅在共振频率处达到最大值,且相位差为$\pi/2$。通过实验数据验证了理论公式(14)和(15)的正确性。

\section{实验原理}
\subsection{阻尼振动}
摆轮-弹簧系统在阻尼力矩$M_r = -\gamma \dot{\theta}$作用下的运动方程为:
\begin{equation}
J \frac{d^2 \theta}{dt^2} = -k \theta - \gamma \frac{d \theta}{dt}
\end{equation}
设阻尼系数$\beta = \gamma/2J$,固有频率$\omega_0 = \sqrt{k/J}$,则方程可改写为:
\begin{equation}
\frac{d^2 \theta}{dt^2} + 2\beta \frac{d \theta}{dt} + \omega_0^2 \theta = 0
\end{equation}

欠阻尼情况下($\beta < \omega_0$)的解为:
\begin{equation}
\theta = \theta_0 e^{-\beta t} \cos(\omega_d t + \varphi_0)
\end{equation}
其中$\omega_d = \sqrt{\omega_0^2 - \beta^2}$为阻尼振动角频率。

\subsection{受迫振动}
在简谐激励$\omega_0^2 A_D \cos(\omega t)$作用下,运动方程为:
\begin{equation}
\frac{d^2 \theta}{dt^2} + 2\beta \frac{d\theta}{dt} + \omega_0^2 \theta = \omega_0^2 A_D \cos(\omega t)
\end{equation}
稳态解为:
\begin{equation}
\theta = \theta_m \cos(\omega t - \varphi)
\end{equation}
其中振幅和相位差分别为:
\begin{equation}
\theta_m = \frac{\omega_0^2 A_D}{\sqrt{(\omega_0^2 - \omega^2)^2 + (2\beta \omega)^2}}
\end{equation}
\begin{equation}
\varphi = \arctan \frac{2\beta \omega}{\omega_0^2 - \omega^2}
\end{equation}

\section{实验仪器及实验步骤}
\subsection{实验仪器}
\begin{itemize}
\item 波耳共振仪主机(含摆轮、弹簧系统)
\item 永久磁铁(用于提供可调阻尼)
\item 步进电机及偏心轮(用于产生受迫振动)
\item 光电门1和2(用于测量振幅和相位差)
\item 有机玻璃转盘(带有角度刻度)
\item 闪光灯装置(用于相位差测量)
\item 数字显示屏(显示周期、振幅等参数)
\end{itemize}

\subsection{实验步骤}
\subsubsection{阻尼振动实验}
\begin{enumerate}
\item 取下永久磁铁,测量最小阻尼状态
\item 拨动摆轮至$150^\circ \sim 180^\circ$后释放
\item 记录振幅$\theta_i$和周期$T_{di}$,共200组数据
\item 安装磁铁,分别设置$s=20$mm和最大距离,重复测量
\item 对$\ln\theta_n$-$n$进行直线拟合,计算$\beta$和$Q$
\end{enumerate}

\subsubsection{受迫振动实验}
\begin{enumerate}
\item 开启电机,确保阻尼足够大
\item 调节强迫力周期,在$0.93T_0 \sim 1.07T_0$范围内取15个点
\item 每个频率点等待系统稳定后,记录$\theta_m$和$\varphi$
\item 绘制幅频和相频特性曲线
\item 从曲线计算共振频率$\omega_r$和品质因数$Q$
\end{enumerate}

\subsubsection{瞬态过程观测}
\begin{enumerate}
\item 设置阻尼$s=20$mm,关闭电机使摆轮静止
\item 以$\omega_0$启动电机,记录振幅随时间变化
\item 绘制瞬态过程曲线,与理论值比较
\end{enumerate}

\section{实验数据处理}

\subsection{阻尼振动部分}
通过线性拟合得到三种阻尼状态下的参数:

\begin{itemize}
\item 无阻尼振动:斜率 = -0.00736,截距 = 5.16709
\item 弱阻尼振动:斜率 = -0.10051,截距 = 5.16337
\item 强阻尼振动:斜率 = -0.14347,截距 = 5.18538
\end{itemize}

\begin{figure}[H]
\centering
\includegraphics[width=0.8\linewidth]{plots/A_without_fit.png}
\caption{无阻尼振动振幅对数线性拟合}
\end{figure}

\begin{figure}[H]
\centering
\includegraphics[width=0.8\linewidth]{plots/A_weak_fit.png}
\caption{弱阻尼振动振幅对数线性拟合}
\end{figure}

\begin{figure}[H]
\centering
\includegraphics[width=0.8\linewidth]{plots/A_strong_fit.png}
\caption{强阻尼振动振幅对数线性拟合}
\end{figure}

\subsection{受迫振动部分}
绘制了幅频特性和相频特性曲线:

\begin{figure}[H]
\centering
\includegraphics[width=0.8\linewidth]{plots/B_weak_amplitude.png}
\caption{弱阻尼幅频特性曲线}
\end{figure}

\begin{figure}[H]
\centering
\includegraphics[width=0.8\linewidth]{plots/B_weak_phase.png}
\caption{弱阻尼相频特性曲线}
\end{figure}

\begin{figure}[H]
\centering
\includegraphics[width=0.8\linewidth]{plots/B_strong_amplitude.png}
\caption{强阻尼幅频特性曲线}
\end{figure}

\begin{figure}[H]
\centering
\includegraphics[width=0.8\linewidth]{plots/B_strong_phase.png}
\caption{强阻尼相频特性曲线}
\end{figure}

\subsection{瞬态过程部分}
\begin{figure}[H]
\centering
\includegraphics[width=0.8\linewidth]{plots/C_instant_amplitude.png}
\caption{瞬态过程振幅变化曲线}
\end{figure}

\section{分析讨论}

\begin{enumerate}
\item 从阻尼振动部分的线性拟合结果可以看出:
\begin{itemize}
\item 无阻尼振动的斜率绝对值最小,说明阻尼系数最小
\item 强阻尼振动的斜率绝对值最大,说明阻尼系数最大
\item 截距值相近,反映了初始振幅的相似性
\end{itemize}

\item 受迫振动部分的幅频特性曲线显示:
\begin{itemize}
\item 弱阻尼时存在明显的共振峰
\item 强阻尼时共振峰不明显,振幅变化平缓
\item 相频特性曲线在共振频率附近发生快速变化
\end{itemize}

\item 瞬态过程分析:
\begin{itemize}
\item 振幅随时间逐渐增大并趋于稳定
\item 达到稳态所需时间与阻尼大小相关
\item 实验结果与理论预测趋势一致
\end{itemize}

\item 品质因数Q的计算:
\begin{itemize}
\item 通过不同方法计算的Q值应进行比较
\item 弱阻尼时的Q值明显大于强阻尼
\item 不同计算方法的结果存在一定差异,需分析误差来源
\end{itemize}

\item 误差分析:
\begin{itemize}
\item 测量系统误差
\item 环境干扰影响
\item 数据处理过程中的近似引入的误差
\item 实验操作中的随机误差
\end{itemize}
\end{enumerate}

\end{document}