\documentclass[UTF8]{ctexart}
\usepackage{geometry}
\geometry{a4paper, margin=1in}
\usepackage{amsmath}
\usepackage{graphicx}
\usepackage{float}
\usepackage{enumitem}
\usepackage{booktabs}

\title{阻尼与受迫振动}
\author{}
\date{}

\begin{document}

\maketitle

\section{摘要}
本实验使用波耳共振仪研究阻尼振动和受迫振动的基本规律。通过调整永久磁铁的位置改变阻尼系数$\beta$,测量不同阻尼状态下摆轮的振幅衰减曲线,计算阻尼系数和品质因数$Q$;在受迫振动实验中,改变外激励频率测量幅频特性和相频特性曲线,观察共振现象。实验结果表明:阻尼振动振幅随时间呈指数衰减,衰减速率与阻尼系数相关;受迫振动达到稳态时,系统以激励频率振动,且在共振频率附近振幅最大、相位差为$\pi/2$。通过实验数据验证了理论公式(6)、(14)和(15)的正确性。

\section{实验原理}
\subsection{阻尼振动}
摆轮-弹簧系统运动方程为:
\[ J\frac{d^2\theta}{dt^2} = -k\theta - \gamma\frac{d\theta}{dt} \]
整理得标准形式:
\[ \frac{d^2\theta}{dt^2} + 2\beta\frac{d\theta}{dt} + \omega_0^2\theta = 0 \]
其中$\beta=\gamma/2J$为阻尼系数,$\omega_0=\sqrt{k/J}$为固有频率。

欠阻尼状态($\beta<\omega_0$)的解为:
\[ \theta = \theta_0 e^{-\beta t}\cos(\omega_d t + \varphi_0) \]
振幅对数衰减满足线性关系:
\[ \ln\theta_n = \ln\theta_0 - \beta t_0 - n(\beta T_d) \]

品质因数定义为:
\[ Q = 2\pi\frac{E}{|\Delta E|} \approx \frac{\omega_0}{2\beta} \]

\subsection{受迫振动}
在外激励$A_D\cos\omega t$作用下,运动方程为:
\[ \frac{d^2\theta}{dt^2} + 2\beta\frac{d\theta}{dt} + \omega_0^2\theta = \omega_0^2 A_D\cos\omega t \]

稳态解为:
\begin{align*}
\theta_m &= \frac{\omega_0^2 A_D}{\sqrt{(\omega_0^2-\omega^2)^2 + (2\beta\omega)^2}} \\
\varphi &= \arctan\frac{2\beta\omega}{\omega_0^2-\omega^2}
\end{align*}

共振时($\omega=\sqrt{\omega_0^2-2\beta^2}$):
\[ \theta_{max} = \frac{\omega_0^2 A_D}{2\beta\sqrt{\omega_0^2-\beta^2}}, \quad \varphi_r = \arctan\frac{\sqrt{\omega_0^2-2\beta^2}}{\beta} \]

\section{实验仪器及实验步骤}
\subsection{实验仪器}
\begin{itemize}
\item 波耳共振仪主机(含摆轮、弹簧系统)
\item 光电门1(测量振幅和周期)
\item 光电门2(测量相位差)
\item 可调阻尼永久磁铁组件
\item 步进电机激励系统(含偏心轮、连杆)
\item 有机玻璃转盘(带角度刻度)
\item 闪光灯相位测量装置
\item 数字显示面板(周期/振幅/相位差)
\end{itemize}

\subsection{实验步骤}
\subsubsection{阻尼振动测量}
\begin{enumerate}
\item 关闭电机,取下磁铁,调整光电门位置
\item 手动拨动摆轮至$150^\circ$后释放,记录振幅$\theta_i$和周期$T_{di}$
\item 重复测量200组数据,用$\ln\theta_n$-$n$拟合求$\beta$
\item 安装磁铁,分别设置$s=20$mm和最大距离,重复测量
\item 计算各阻尼状态的$Q$值
\end{enumerate}

\subsubsection{受迫振动测量}
\begin{enumerate}
\item 开启电机,设置中等阻尼($s=20$mm)
\item 调节激励频率$\omega$,每个频率点等待系统稳定
\item 记录稳态振幅$\theta_m$和相位差$\varphi$
\item 在$0.93T_0$-$1.07T_0$范围内取15个数据点
\item 更换阻尼条件重复测量
\item 绘制幅频/相频特性曲线
\end{enumerate}

\subsubsection{瞬态过程观测}
\begin{enumerate}
\item 设置$\omega=\omega_0$,从静止状态启动电机
\item 逐周期记录振幅变化直至稳态
\item 绘制振幅-时间曲线
\end{enumerate}

\section{实验数据处理}
\subsection{阻尼振动数据处理}
通过线性拟合得到弱阻尼和强阻尼状态下的阻尼系数:
\begin{itemize}
    \item 弱阻尼状态:$\beta = 0.1005 \text{s}^{-1}$
    \item 强阻尼状态:$\beta = 0.1435 \text{s}^{-1}$
\end{itemize}

根据弱阻尼数据计算固有角频率:
\[ \omega_0 = \sqrt{\left(\frac{2\pi}{1.5691}\right)^2 + \beta^2} = 4.007 \text{rad/s} \]

品质因数计算结果:
\begin{itemize}
    \item 弱阻尼:$Q = \frac{\omega_0}{2\beta} = 19.94$
    \item 强阻尼:$Q = \frac{\omega_0}{2\beta} = 13.96$
\end{itemize}

\begin{figure}[H]
    \centering
    \includegraphics[width=0.8\linewidth]{plots/weak_damping_fit.png}
    \caption{弱阻尼状态下振幅对数与时间的关系}
    \label{fig:weak_damping}
\end{figure}

\begin{figure}[H]
    \centering
    \includegraphics[width=0.8\linewidth]{plots/strong_damping_fit.png}
    \caption{强阻尼状态下振幅对数与时间的关系}
    \label{fig:strong_damping}
\end{figure}

\subsection{受迫振动数据处理}
幅频特性曲线如图\ref{fig:amplitude_frequency}所示,共振频率$\omega_r \approx \omega_0$。

\begin{figure}[H]
    \centering
    \includegraphics[width=0.8\linewidth]{plots/weak_amplitude_frequency.png}
    \caption{弱阻尼状态下的幅频特性曲线}
    \label{fig:weak_amp_freq}
\end{figure}

\begin{figure}[H]
    \centering
    \includegraphics[width=0.8\linewidth]{plots/strong_amplitude_frequency.png}
    \caption{强阻尼状态下的幅频特性曲线}
    \label{fig:strong_amp_freq}
\end{figure}

\subsection{瞬态过程分析}
瞬态过程振幅随时间变化如图\ref{fig:transient}所示。

\begin{figure}[H]
    \centering
    \includegraphics[width=0.8\linewidth]{plots/transient_amplitude.png}
    \caption{瞬态过程振幅随时间变化}
    \label{fig:transient}
\end{figure}

\section{分析讨论}
\begin{enumerate}
    \item 阻尼系数测量结果表明,随着阻尼增大,$\beta$值增大,品质因数$Q$减小,与理论预期一致。
    
    \item 幅频特性曲线显示,弱阻尼时共振峰尖锐,强阻尼时共振峰宽平,符合理论预测。通过半功率带宽法计算得到的$Q$值与阻尼振动结果基本吻合。
    
    \item 瞬态过程曲线显示振幅随时间逐渐增大至稳态,与理论推导的瞬态过程公式相符。稳态后电机平均输入功率可通过$P = \frac{1}{2}k\theta_m^2\omega_0/Q$计算。
    
    \item 实验误差主要来源于:振幅读数误差、阻尼调节不均匀、环境干扰等。建议改进测量方法,提高读数精度,减少环境振动干扰。
\end{enumerate}

\end{document}