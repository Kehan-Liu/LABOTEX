\documentclass[UTF8]{ctexart}
\usepackage{geometry}
\geometry{margin=1in}
\usepackage{amsmath}
\usepackage{float}
\usepackage{enumitem}
\usepackage{graphicx}

\title{阻尼与受迫振动}
\author{}
\date{}

\begin{document}

\maketitle

\section{摘要}
本实验使用波耳共振仪研究了阻尼振动和受迫振动的特性。通过调整永久磁铁的位置改变阻尼系数,测量了不同阻尼状态下的振动衰减曲线,计算了阻尼系数$\beta$和品质因数$Q$。在受迫振动实验中,通过改变激励源的频率,测量了系统的幅频特性和相频特性曲线,观察了共振现象。实验结果表明:1) 阻尼振动振幅随时间呈指数衰减,衰减速率与阻尼系数相关;2) 受迫振动的稳态振幅和相位差随激励频率变化,在共振频率附近达到最大值;3) 共振时系统的振动能量转换效率最高。

\section{实验原理}
\subsection{阻尼振动}
摆轮-弹簧系统的运动方程为:
\[ J\frac{d^2\theta}{dt^2} = -k\theta - \gamma\frac{d\theta}{dt} \]
整理得:
\[ \frac{d^2\theta}{dt^2} + 2\beta\frac{d\theta}{dt} + \omega_0^2\theta = 0 \]
其中$\beta=\gamma/2J$为阻尼系数,$\omega_0=\sqrt{k/J}$为固有角频率。

欠阻尼($\beta<\omega_0$)解为:
\[ \theta = \theta_0 e^{-\beta t}\cos(\omega_d t + \varphi_0) \]
其中$\omega_d=\sqrt{\omega_0^2-\beta^2}$为阻尼振动角频率。

通过测量振幅衰减曲线,利用对数衰减法可求$\beta$:
\[ \ln\theta_n = \ln\theta_0 - \beta t_0 - n(\beta T_d) \]

品质因数$Q$定义为:
\[ Q = \frac{\omega_0}{2\beta} \]

\subsection{受迫振动}
在简谐激励$\omega_0^2A_D\cos(\omega t)$作用下,运动方程为:
\[ \frac{d^2\theta}{dt^2} + 2\beta\frac{d\theta}{dt} + \omega_0^2\theta = \omega_0^2A_D\cos(\omega t) \]

稳态解为:
\[ \theta = \theta_m\cos(\omega t - \varphi) \]
其中振幅和相位差为:
\[ \theta_m = \frac{\omega_0^2A_D}{\sqrt{(\omega_0^2-\omega^2)^2+(2\beta\omega)^2}} \]
\[ \varphi = \arctan\frac{2\beta\omega}{\omega_0^2-\omega^2} \]

共振时($\omega=\omega_r$):
\[ \omega_r = \sqrt{omega_0^2-2\beta^2} \]
\[ \theta_{max} = \frac{\omega_0^2A_D}{2\beta\sqrt{\omega_0^2-\beta^2}} \]
\[ \varphi = \frac{\pi}{2} \]

\section{实验仪器及实验步骤}
\subsection{实验仪器}
\begin{itemize}
    \item 波耳共振仪主机
    \item 摆轮-弹簧振动系统
    \item 可调阻尼的永久磁铁装置
    \item 光电门1(测量振幅和周期)
    \item 光电门2(测量相位差)
    \item 步进电机激励源
    \item 有机玻璃转盘(带角度刻度)
    \item 闪光灯相位测量装置
    \item 数据采集和显示系统
\end{itemize}

\subsection{实验步骤}
\subsubsection{阻尼振动实验}
\begin{enumerate}
    \item 调整仪器:确认电机开关关闭,取下永久磁铁,检查摆轮自由摆动无摩擦
    \item 测量最小阻尼状态:拨动摆轮至$150^\circ\sim180^\circ$,记录振幅衰减曲线(约200个周期)
    \item 安装磁铁:设置$s=20$mm,重复振幅测量
    \item 最大阻尼状态:将磁铁调至最低位置,重复测量
    \item 数据处理:用对数衰减法计算各状态的$\beta$和$Q$值
\end{enumerate}

\subsubsection{受迫振动实验}
\begin{enumerate}
    \item 开启电机:确认阻尼足够大($s=20$mm)
    \item 幅频特性测量:
    \begin{itemize}
        \item 调节激励频率从$0.93T_0$到$1.07T_0$
        \item 每个频率点等待系统稳定后记录$\theta_m$和$\varphi$
        \item 在共振点附近加密测量点
    \end{itemize}
    \item 改变阻尼状态(如$s=10$mm),重复测量
    \item 从幅频曲线计算$Q$值,与阻尼振动结果比较
\end{enumerate}

\subsubsection{瞬态过程观测}
\begin{enumerate}
    \item 设置激励频率为$\omega_0$,阻尼$s=20$mm
    \item 从静止状态启动电机,记录振幅增长过程
    \item 绘制振幅-时间曲线,分析瞬态特性
\end{enumerate}

\end{document}