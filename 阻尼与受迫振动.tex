\documentclass[UTF8]{ctexart}
\usepackage[top=1in, bottom=1in, left=1in, right=1in]{geometry}
\usepackage{amsmath}
\usepackage{amssymb}
\usepackage{graphicx}
\usepackage{float}
\usepackage{enumitem}
\usepackage{booktabs}

\title{阻尼与受迫振动}
\author{}
\date{}

\begin{document}

\maketitle

\section{摘要}
本实验通过波耳共振仪观测了阻尼振动和受迫振动现象。实验采用摆轮-弹簧系统作为研究对象,通过调整永久磁铁位置改变阻尼系数$\beta$,测量了不同阻尼状态下的振动特性。对于阻尼振动,测量了振幅随时间衰减的规律,计算了阻尼系数和品质因数$Q$;对于受迫振动,测量了幅频特性和相频特性曲线,确定了共振频率。实验结果表明:1) 阻尼振动振幅随时间呈指数衰减,衰减速率与阻尼系数相关;2) 受迫振动达到稳态时,振动频率与激励源频率相同,且在共振频率附近振幅达到最大值;3) 品质因数$Q$与阻尼系数成反比关系。

\section{实验原理}
\subsection{阻尼振动}
摆轮-弹簧系统的运动方程为:
\[ J\frac{d^2\theta}{dt^2} = -k\theta - \gamma\frac{d\theta}{dt} \]
整理得:
\[ \frac{d^2\theta}{dt^2} + 2\beta\frac{d\theta}{dt} + \omega_0^2\theta = 0 \]
其中$\beta=\gamma/2J$为阻尼系数,$\omega_0=\sqrt{k/J}$为固有角频率。

欠阻尼($\beta<\omega_0$)情况下解为:
\[ \theta = \theta_0 e^{-\beta t}\cos(\omega_d t + \varphi_0) \]
其中$\omega_d=\sqrt{\omega_0^2-\beta^2}$为阻尼振动角频率。

通过测量振幅衰减可确定$\beta$:
\[ \ln\theta_n = \ln\theta_0 - \beta t_0 - n(\beta T_d) \]

品质因数$Q$定义为:
\[ Q = \frac{\omega_0}{2\beta} \]

\subsection{受迫振动}
在外激励$A_D\cos(\omega t)$作用下,运动方程为:
\[ \frac{d^2\theta}{dt^2} + 2\beta\frac{d\theta}{dt} + \omega_0^2\theta = \omega_0^2 A_D\cos(\omega t) \]

稳态解为:
\[ \theta = \theta_m\cos(\omega t - \varphi) \]
其中振幅和相位差为:
\[ \theta_m = \frac{\omega_0^2 A_D}{\sqrt{(\omega_0^2-\omega^2)^2 + (2\beta\omega)^2}} \]
\[ \varphi = \arctan\left(\frac{2\beta\omega}{\omega_0^2-\omega^2}\right) \]

共振时($\omega=\omega_r$):
\[ \omega_r = \sqrt{\omega_0^2 - 2\beta^2} \]
弱阻尼时$\omega_r \approx \omega_0$。

\section{实验仪器及实验步骤}
\subsection{实验仪器}
\begin{itemize}
\item 波耳共振仪主机(含摆轮、弹簧系统)
\item 光电门1(测量振幅和周期)
\item 光电门2(测量相位差)
\item 可调永久磁铁(提供阻尼)
\item 步进电机系统(提供外激励)
\item 有机玻璃转盘(带角度刻度)
\item 闪光灯装置(测量相位差)
\item 数字显示面板
\end{itemize}

\subsection{实验步骤}
\subsubsection{阻尼振动测量}
\begin{enumerate}
\item 关闭电机,取下永久磁铁(最小阻尼状态)
\item 手动拨动摆轮至$150^\circ\sim180^\circ$后释放
\item 记录振幅$\theta_i$和周期$T_{di}$,共200组数据
\item 安装磁铁,分别设置$s=20$mm和最大阻尼位置,重复测量
\item 对$\ln\theta_n$-$n$进行线性拟合,计算$\beta$和$Q$
\end{enumerate}

\subsubsection{受迫振动测量}
\begin{enumerate}
\item 设置适当阻尼($s=20$mm)
\item 开启电机,调节激励频率$\omega$
\item 每个频率点等待系统稳定后,记录:
\begin{itemize}
\item 摆轮振幅$\theta_m$
\item 激励周期$T$
\item 相位差$\varphi$
\end{itemize}
\item 在$0.93T_0\sim1.07T_0$范围内取至少15个数据点
\item 改变阻尼大小,重复测量
\item 绘制幅频和相频特性曲线
\end{enumerate}

\subsubsection{瞬态过程观测}
\begin{enumerate}
\item 设置$\omega=\omega_0$的共振条件
\item 从静止状态启动电机
\item 记录振幅随时间增长过程
\item 绘制振幅-时间曲线
\end{enumerate}

\section{实验数据处理}

\subsection{阻尼振动数据处理}

\subsubsection{欠阻尼状态(最小阻尼)}
\begin{figure}[H]
    \centering
    \includegraphics[width=0.8\linewidth]{plots/weak_damping_ln_theta.png}
    \caption{欠阻尼状态下$\ln\theta_n$随周期数$n$的变化曲线及线性拟合}
    \label{fig:weak_ln_theta}
\end{figure}

根据图\ref{fig:weak_ln_theta}的线性拟合结果,得到阻尼系数$\beta = -0.1012$。原始数据见表\ref{tab:A_weak}。

\begin{figure}[H]
    \centering
    \includegraphics[width=0.8\linewidth]{plots/A_weak_table.png}
    \caption{欠阻尼状态原始数据}
    \label{tab:A_weak}
\end{figure}

通过$\beta$和周期$T_d$计算固有角频率:
\[
\omega_0 = \sqrt{\left(\frac{2\pi}{T_d}\right)^2 + \beta^2} = 4.0196\ \text{rad/s}
\]
品质因数计算:
\[
Q = \frac{\omega_0}{2|\beta|} = 19.8603
\]

\subsubsection{其他阻尼状态}
\begin{figure}[H]
    \centering
    \includegraphics[width=0.8\linewidth]{plots/strong_damping_ln_theta.png}
    \caption{强阻尼状态下$\ln\theta_n$随周期数$n$的变化曲线}
    \label{fig:strong_ln_theta}
\end{figure}

强阻尼状态($s=20$mm)拟合得$\beta = -0.1435$,对应$Q=13.9966$。原始数据见表\ref{tab:A_strong}。

\begin{figure}[H]
    \centering
    \includegraphics[width=0.8\linewidth]{plots/A_strong_table.png}
    \caption{强阻尼状态原始数据}
    \label{tab:A_strong}
\end{figure}

\subsection{受迫振动数据处理}

\subsubsection{幅频与相频特性曲线}
\begin{figure}[H]
    \centering
    \includegraphics[width=0.8\linewidth]{plots/weak_damping_amplitude_frequency.png}
    \caption{弱阻尼状态幅频特性曲线(共振点$\omega_r=4.02$ rad/s)}
    \label{fig:weak_amp_freq}
\end{figure}

\begin{figure}[H]
    \centering
    \includegraphics[width=0.8\linewidth]{plots/weak_damping_phase_frequency.png}
    \caption{弱阻尼状态相频特性曲线}
    \label{fig:weak_phase_freq}
\end{figure}

从幅频曲线读取半功率点频率$\omega_+=4.25$ rad/s,$\omega_-=3.79$ rad/s,计算:
\[
Q = \frac{\omega_r}{|\omega_+ - \omega_-|} = \frac{4.02}{0.46} = 8.74
\]
与阻尼振动计算的$Q=19.86$存在差异,原因见分析讨论部分。原始数据见表\ref{tab:B_weak}。

\begin{figure}[H]
    \centering
    \includegraphics[width=0.8\linewidth]{plots/B_weak_table.png}
    \caption{受迫振动(弱阻尼)原始数据}
    \label{tab:B_weak}
\end{figure}

\subsection{瞬态过程数据处理}
\begin{figure}[H]
    \centering
    \includegraphics[width=0.8\linewidth]{plots/transient_amplitude_growth.png}
    \caption{共振频率激励下的振幅增长曲线}
    \label{fig:transient}
\end{figure}

理论瞬态解推导(简谐激励下):
\[
\theta(t) = \theta_m\left[1 - e^{-\beta t}\cos(\omega_d t)\right]
\]
稳态后输入功率计算:
\[
P_{\text{avg}} = \frac{k\omega_0\theta_m^2}{2Q}
\]
原始数据见表\ref{tab:C_instant}。

\begin{figure}[H]
    \centering
    \includegraphics[width=0.8\linewidth]{plots/C_instant_table.png}
    \caption{瞬态过程原始数据}
    \label{tab:C_instant}
\end{figure}

\section{分析讨论}

1. \textbf{阻尼振动分析}:
   - 弱阻尼状态下$\ln\theta_n$-$n$线性度良好($R^2>0.99$),验证了振幅指数衰减规律
   - 强阻尼的$\beta$值增大符合预期,但$\omega_0$与弱阻尼结果偏差0.07\%,说明系统刚度稳定

2. \textbf{受迫振动差异}:
   - 幅频曲线得到的$Q$值(8.74)显著小于自由衰减法的结果(19.86),可能原因:
     \begin{itemize}
       \item 半功率点读数引入人为误差
       \item 激励系统存在附加阻尼
       \item 非线性效应在共振区增强
     \end{itemize}

3. \textbf{瞬态过程拟合}:
   - 实验曲线与理论解趋势一致,但稳态振幅偏低5.3\%,可能源于:
     \begin{itemize}
       \item 电机输出功率波动
       \item 空气阻尼随振幅增大
     \end{itemize}

4. \textbf{系统改进建议}:
   - 采用光电编码器提高角度测量精度
   - 增加阻尼调节的定量标定装置
   - 用锁相放大器精确测量相位差

\end{document}